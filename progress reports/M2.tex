\documentclass{article}
\usepackage[colorlinks=true]{hyperref}
\usepackage{geometry}
\usepackage{fancyhdr}
\usepackage{palatino}
\usepackage{titlesec}
\usepackage{pbox}
\usepackage{multicol}
\usepackage{graphicx}
\usepackage{amsmath}
\usepackage{enumitem}

\renewcommand{\baselinestretch}{1.15}
\geometry{margin=1in}
\geometry{headheight=2in}
\geometry{top=2in}
\titlespacing\section{0pt}{12pt plus 2pt minus 2pt}{0pt plus 2pt minus 2pt}
\lhead{}
\rhead{}
\pagestyle{fancy}
\setlength{\parskip}{0.5em}
\date{\today}

\title{CS 145 Milestone 2 - ToMEto}
\author{Jonathan Joo, Matthew Jin, Boyu (Charlie) Tong, Albert Ge}
\chead{%
  {\vbox{%
      \vspace{2mm}
      \large
      Networks: Structure Economics \hfill
      Caltech CMS/CS/EE 145 \hfill \\[1pt]
      CS 145 Milestone 1 - ToMEto \hfill
      \date{\today} \\
    }
  }
}

\begin{document}
\maketitle

\section{Goals}
This week, we wanted to accomplish
\begin{enumerate}
    \item Designing and planning algorithms to process the data.
\end{enumerate}

\section{Progress}

We accomplished two preliminary algorithms this week.

\subsection{Algorithms}
The first algorithm (in \texttt{analyzer.py}) constructed a graph.
\begin{itemize}
    \item Using individual ingredients as nodes, and then ranked the nodes by the number of
recipes it appeared in with adjacent neighbors (ingredients in the same list).
    \item Each edge in the network also had an associated weight, which quantifies
the number of recipes that the two connected ingredients both appear in.
    \item Obtained preliminary results on the most 
important ingredients in the network. Furthermore, used 
these ingredients to determine best-matching \textit{complements}
for a given recipe, which implements naively a ingredient
that could be added into a recipe. 
    \item For 
each ingredient in the recipe, we found their external
ingredient (i.e. outside of the recipe), summing over
the weights of each connection, normalizing by degree of each ingredient.
Additionally,
we ignored the top ten most common ingredients (so as to 
not suggest something obvious).
\end{itemize}
The second algorithm (in \texttt{analyzer\_PMI.py}) constructs a simmilar
graph, however instead of the weights being the number of recipes, 
we used the \textit{Pointwise mutual information} metric, defined by
\[
    PMI(a,b) = \frac{\text{\# recipes containing a and b}}{
    (\text{\# recipes containing only a})(\text{\# recipes containing only b})}
\]

\subsection{Front-end}
Additionally, we
worked more on front-end.
\begin{itemize}
    \item Got a few different pages up. 
    \item Worked on logo design as well as more UI for the landing page, which can be seen in the "landing2" tab. 
    \item Now able to handle input/output on search forms, which can now be integrated fully into our
python scripts. Thus, using this implementation of our search, we can use search algorithms to find the proper
recipes from our database to deliver to a user upon a search input. 
\end{itemize}

\subsection{Engineering Design}

\begin{itemize}
    \item
Began refactoring our initial analyzer into the actual backend
    \item Brainstormed and discussed with Albert on algorithms
\end{itemize}

\section{Contributions}
\textbf{Jonathan Joo:} Front-end, begin linking to back-end

\textbf{Matthew Jin:} Reading on additional literature, helped Albert with 
    choosing algorithms to try

\textbf{Charlie Tong:} Continued work on \texttt{analyzer.py}, implementing
    first algorithm 

\textbf{Albert Ge:} Reading on additional literature, implemented PMI algorithm



\section{Adjustments to plan}

We are still in the stage of developing good algorithms for the network,
which is inline with the M2 goals we had. We will probably continue
to work on determining a good algorithm, and as we reach more
optimal results, begin to seek out interviewers again for 
feedback on the results that we get.

Furthermore, as we are beginning to link front-end with back-end,
we are thinking more about good design principles in engineering.
We took a very sledgehammer approach to get things off the ground and running,
however
as we continue to grow our codebase, we feel that it will be important
to refactor the code to make it more readable, scalable, and less bug-prone.

\end{document}
